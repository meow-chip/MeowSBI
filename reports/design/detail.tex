\documentclass[UTF-8]{ctexart}

\usepackage{listing}
\usepackage{minted}

\title{rCore on MeowV64 - 设计方案}

\author{g04 - 刘晓义}
\date{2020-3}

\begin{document}
\maketitle

\section{课程设计目标}

为 MeowV64 实现 M 态软件,并补全其硬件,使得在其上可以运行 rCore (或 rCore 的裁剪版本)。

\section{课程设计背景}

\subsection{MeowV64 和 RISC-V ISA}
MeowV64 是在上个学期计算机组成原理中,由刘晓义、娄晨曜、申奥共同完成的 RISC-V 64 IMAC 处理器(其中 A 支持不全)。MeowV64 包含 L1 和 L2 两级 Cache,并且支持 SMP。流水线支持多发射乱序执行,但是这个对软件而言是透明的。

RISC-V ISA 对于软件适配系统而言有特别方便的地方,因为区分了 M 态软件和 S 态软件,并且在这两个特权态的界面上有一个接口的参考标准,也就是 SBI 标准。

因此,硬件系统开发者只需要实现适配了自身平台的 M 态软件,就可以支持所有的 S 态应用。

\section{既有社区工作}
  \subsection{OpenSBI}
  OpenSBI 是 RISC-V 社区的一个参考 SBI 实现,有很好的可移植性。它内置了一个 FDT Parser,并且支持全部的 SBI 扩展。硬件系统开发者只需要实现很少的一些元语,就可以将 OpenSBI 移植到自己的系统内。

  因此,一个实现路径是直接使用 OpenSBI,实现平台相关代码,即可完成 M 态软件。

  另一个可行的实现路径是自己编写 Bootloader,然后将 OpenSBI 作为第二级引导项或者一个SBI 调用处理库使用。
\section{实现方案}

  \subsection{实现路径}

  虽然 OpenSBI 比较好用,但是出于造轮子的心态,我们决定重新编写一个 M 态软件。另外的部分动机来自于我们阅读了 OpenSBI 的源代码,发现其中大量代码在实现 Spinlock,Channel 等代码,以及包装 RISC-V 汇编指令。如果使用 Rust 编写,可以大幅简化代码。关于语言选择的讨论,可以具体参考第 \ref{sec:lang} 节。

  具体需要实现的部分包括:

  \begin{itemize}
    \item SBI 接口
    \begin{itemize}
      \item Base
      \item Time, IPI, RFNC (and their legacy conterparts)
    \end{itemize}
    \item FDT parser
  \end{itemize}

  注意到,我们决定不实现 SBI 中的 HSM 扩展。主要原因是这部分平台相关的元语和其他 SBI 接口完全不相交,而且 MeowV64 不支持关闭其中的部分核心。

  当然还有一种可能是实现和 SBI 不兼容的 M 态软件,然后修改 rCore 代码适配新的接口。但是事实上 SBI 的设计非常简洁,而且基本上所有的 S 态软件都要依赖这部分接口,因此设计新接口的意义不大。

  \subsection{语言选择}
  \label{sec:lang}
  我们选择了 Rust 语言,主要考量了其三点特性:
  
  \paragraph{安全边界界定} 注意,这里并不指传统用户态或者操作系统内的安全的定义,因为在 M 态,我们拥有系统的完全权限,做什么都是“合法的”。在这里,我们主要需要注意的是 Undefined Behavior 的边界,以及如何指挥编译器进行优化。例如在刚进入 M 态软件的时候,全部 RAM 内存都应该是无意义的内容。在这时,我们可以用 MaybeUninit 告诉编译器在这段时间不要进行特定的优化,例如预先假定 bool 类型的内存里只能有 0 和 1 两种可能。

  控制并发竞争事实上在 M 态里比较少见,可能包括的就是串口 IO 和远程 FENCE。由于中断控制器的操作一般是原子的,因此这不涉及同步问题。

  \paragraph{高级抽象和外部代码} Pattern Matching 以及 Option,Result 等标准类型会很大程度方便接口的编写。同时,社区有成熟的同步元语、自动类型转换以及 RISC-V 汇编的包装,可以直接使用。

  \subsection{硬件及 SoC}
  MeowV64 及 SoC 本身需要完成部分工作:

  \paragraph{Sv48 页表支持} 目前没有页表支持,这是比较重要的一项。

  \paragraph{中断控制器} 目前虽然支持外部中断,但是没有中断控制器。可以考虑直接用 Xilinx 提供的中断控制器 IP,自己写如果只需要比较简单的功能,推测难度也不太大。

  \paragraph{AMO} 目前支持的 A 扩展指令只有 \texttt{lr} 和 \texttt{sc}。Rust 编译目标需要 ISA 支持 AMO。虽然可以通过 M 态模拟,但是如果通过硬件实现可以复用部分逻辑,并且可以大幅度提高性能。

  \paragraph{面积优化} 目前片上只能勉强放下两个核心,并且频率不能太高。发射阶段和 ROB 的逻辑需要进一步优化,以减小面积占用。

  \subsection{后续工作}
  可能需要进行的包括U-boot的适配工作,这样会比较方便从外设里加载 rCore 的代码。

  如果可能的话,我们会尝试使用上个学期编写的 Rust 前端编译 rCore 的缩减版(完整支持 Rust 的语言特性有很大困难)。

  \section{已完成工作}
  刚刚开始实现适配 Qemu 的 M 态软件,初步目标是在 Qemu 上替代 OpenSBI。
\end{document}
