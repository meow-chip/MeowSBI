\documentclass[UTF-8]{ctexbeamer}
\usetheme{Boadilla}

\usepackage{listing}
\usepackage{minted}

\title{rCore on MeowV64}
\subtitle{Pre-final report}

\author{g04 - 刘晓义}
\date{2020-5}

\begin{document}
\begin{frame}
  \titlepage
\end{frame}

\begin{frame}
  \frametitle{What was this?}

  向 MeowV64 添加足够的功能,可以实现多核并行,并实现 MeowSBI 作为其 M 态软件,在其上运行 rCore.

  \begin{description}
    \item[rCore] \url{https://github.com/rcore-os/rCore}
    \item[MeowV64] 某个 RISC-V 处理器: \url{https://github.com/meow-chip/MeowV64}
    \item[MeowSBI] 某个 RISC-V 的 M 态软件: \url{https://github.com/meow-chip/MeowSBI}
  \end{description}

  \pause

  \begin{center}
    \begin{tabular}{|ll|} \hline
      OS & \textbf{rCore}\\
      \hline
      Firmware & \textbf{Meow / OpenSBI} \\
      \hline
      Processor & \textbf{MeowV64}\\
      \hline
    \end{tabular}
  \end{center}
\end{frame}
\begin{frame}
  \frametitle{About the processor}

  \begin{itemize}
    \item 添加了多核支持,包括 A-extension, L2 Cache + Coherence protocol
    \item 添加了 S 态和 U 态
    \item 添加了页表以及相关的 PTW 接口
    \item 添加了 CLINT,实现了跨核软件中断和时钟中断
    \item FDT
  \end{itemize}
\end{frame}

\begin{frame}
  \frametitle{About the software}

  实现了 rCore 需要的 SBI 功能

  \begin{itemize}
    \item relocate FDT
    \item CLINT、串口初始化
    \item 提供部分 SBI call
    \begin{itemize}
      \item BASE extension
      \item TIME extension
      \item Console(Put|Get)Char
      \item SetTimer
      \item Shutdown
    \end{itemize}
  \end{itemize}
\end{frame}

\begin{frame}
  \frametitle{About other stuff}

  \begin{itemize}
    \item 找到了 QEMU 的一个和 timer 相关的 bug
    \item 修复了 rCore 的一点 Bug
  \end{itemize}

  \pause

  \vspace{1em}
  \textbf{一些尝试}
  \begin{itemize}
    \item 使用尽量少的裸汇编
    \item 使用尽量多的静态派发
  \end{itemize}
\end{frame}

\begin{frame}
  \frametitle{Status \& What's next}

  现在,OpenSBI 和 MeowSBI 都可以:
  \begin{itemize}
    \item 在\textbf{单核 MeowV64 硬件} 上执行简单的 Payload
    \item 在\textbf{多核 MeowV64 仿真} 中正确初始化 rCore
    \item 在\textbf{多核 QEMU} 中基本完整正确运行 rCore 的用户态
  \end{itemize}

  \pause
  \vspace{1em}

  还没搞成的事情:
  \begin{itemize}
    \item MeowSBI 的 Release 版本还跑不动
    \item FPGA 片上空间仍然紧缺
    \item PLIC 以及板上的外设的连接
    \begin{itemize}
      \item PLIC 有模块化的实现,可以作为总线上的设备
      \item 需要更新 FDT
    \end{itemize}
  \end{itemize}
\end{frame}

\begin{frame}
  \frametitle{Q \& A}

  Thanks!
\end{frame}

\end{document}
